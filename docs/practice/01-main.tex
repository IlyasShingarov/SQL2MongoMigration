\chapter{Основная часть}

В данном разделе формализуется постановка задачи, 
описываются средства программной реализации метода 
миграции данных из реляционной в документо-ориентированную базу данных.
Проводится тестирование метода и предоставляются его результаты.
Описывается взаимодействие пользователя с интерфейсом ПО, реализующим метод.



\section{Постановка задачи}
Формализуем постановку задачи в виде IDEF0-диаграммы.
На рисунке \ref{img:diploma-A0} представлена IDEF0-диаграмма метода
миграции данных из реляционной в документо-ориентированную базу данных нулевого уровня.

\includeimage{diploma-A0}{f}{h}{\textwidth}{IDEF0-диаграмма нулевого уровня}

\clearpage

\section{Входные и выходные данные}
В качестве входных данных разработанное ПО принимает:
\begin{itemize}[label=---]
    \item подключение к РСУБД PostgreSQL;
    \item подключение к документо-ориентированной СУБД MongoDB;
    \item название схемы;
    \item информация о предметной области в виде выбранных таблиц и кратности отношений между сущностями;
\end{itemize}

Выходными данными являются мигрированные данные в MongoDB.


\section{Используемые средства}
Для реализации разрабатываемого метода был выбран язык Java версии 17~\cite{openjdk-17}.
Выбор языка обусловлен наличием опыта разработки и нативной поддержкой протокола JDBC,
что позволяет взаимодействие с большинством РСУБД посредством JDBC-драйвера~\cite{jdbc-api}.

В разработанном ПО были использованы следующие библиотеки и фреймворки:
\begin{itemize}[label=---]
    \item Spring Framework (Spring Boot)~\cite{spring-boot} --- фреймворк реализующий контейнер инверсии зависимостей, 
    используется для связывания модулей приложения и конфигурации;
    \item Spring Data JDBC~\cite{spring-data-jdbc} --- модуль интегрирующий JDBC в Spring Boot и
    предоставляющий верхнеуровневый интерфейс для обращения к базе данных;
    \item Spring Data MongoDB~\cite{spring-data-mongodb} --- модуль предоставляющий интерфейс обёртку для драйвера MongoDB;
    \item Spring Shell~\cite{spring-shell} --- модуль предоставляющий функциональность для разработки консольных интерфейсов;
    \item Jackson~\cite{jackson-databind} --- библиотека предоставляющая интерфейс для работы с JSON-объектами;
    \item pg\_stat\_statements~\cite{pg-stat-statements} --- модуль PostgreSQL логирующий запросы к СУБД.
\end{itemize}


\section{Требования к разрабатываемому ПО}
Разработанное ПО, должно предоставлять:
\begin{itemize}[label=---]
    \item возможность подключения к реляционной и документо-ориентированной базе данных;
    \item возможность выбора схемы и таблиц, которые будут мигрированы;
    \item возможность ввода кратности отношений между сущностями;
    \item возможность переноса данных из PostgreSQL в MongoDB;
\end{itemize}


\section{Структура разработанного ПО}

\section{Тестирование}

\section{Пользовательский интерфейс}