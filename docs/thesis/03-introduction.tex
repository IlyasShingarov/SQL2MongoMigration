\chapter*{ВВЕДЕНИЕ}
\addcontentsline{toc}{chapter}{ВВЕДЕНИЕ}

Крупные компании в наше время имеют высокую потребность в сборе
и анализе информации. В частности, множество компаний имеют большое
число клиентов и как следствие, с ростом числа клиентов растет объем
транзакционных данных. Появляется потребность в том, чтобы проводить
быструю аналитику для формирования, например, маркетинговых кампаний,
однако зачастую они упираются в производительность СУБД.

Наиболее популярным решением для хранения данных выступают реляционные базы данных. Они заняли большую часть рынка и являются наиболее
оптимальным решением при малом или среднем объеме данных~\cite{db_rating}. 
Однако, при росте объемов данных, достоинства РСУБД начинают их ограничивать.
И с повсеместным распространением облачных и распределенных технологий,
появляется потребность в выборе альтернативного решения для хранения данных.
Как следствие, возникает потребность в переносе данных в новое хранилище.


Цель работы – разработать метод миграции данных из реляционной в
документо-ориентированную базу данных с
использованием частотного и семантического
анализа.

Для достижения поставленной цели необходимо решить следующие
задачи:
\begin{itemize}[label=---]
    \item рассмотреть подходы к миграции данных из реляционных в документо-ориентированные
    базы данных и провести сравнение методов миграции данных;
    \item разработать метод миграции данных из реляционной в документо-ориентированную базу данных с
    использованием частотного и семантического анализа;
    \item спроектировать и реализовать ПО, демонстрирующее работу метода;
    \item провести <\dots>.
\end{itemize}

