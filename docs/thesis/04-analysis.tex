\chapter{Аналитический раздел}

В данном разделе производится анализ предметной области, 
дается краткая характеристика различных типов баз данных,
приводятся основные понятия и излагаются области применения миграции баз данных. 
Приводится описание и сравнение методов миграции данных 
из реляционной в документо-ориентированную базу данных. 
Производится постановка задачи в виде IDEF0-диаграммы.


\section{Базы данных}

База данных (БД) -- совокупность данных, организованных в соответствии
с концептуальной структурой, описывающей характеристики этих данных и
взаимоотношения между ними, которая поддерживать одну 
или более областей применения.

Система управления базами данных (СУБД) -- комплекс программных средств,
который предоставляет возможность обращаться к базе данных 
для некоторой группы пользователей.
 
Задачи СУБД:
\begin{itemize}[label=---]
    \item управление данными во внешней памяти;
    \item управление транзакциями;
    \item журналирование;
    \item поддержка языка запросов;
\end{itemize}

СУБД можно классифицировать по используемой для организации хранения модели данных.
Среди разных моделей данных существуют такие как:
иерархическая, сетевая, объектно-ориентированная, реляционная, ассоциативная и другие.

Реляционная модель данных, является наиболее широко используемой.
На таблице~\ref{tab:db_ranking} представлен рейтинг СУБД по популярности на момент
марта 2023 года с указанием используемой модели данных~\cite{db_rating}.

\clearpage

\begin{table}[h!]
    \centering
    \caption{Рейтинг популярности СУБД}
    \label{tab:db_ranking}
    \begin{tabular}{|l|l|l|l|}
        \hline
        \multicolumn{1}{|c|}{Позиция} & \multicolumn{1}{c|}{СУБД} & \multicolumn{1}{c|}{Модель данных} & \multicolumn{1}{c|}{Рейтинг} \\ \hline
        1 & Oracle     & Реляционная               & 1232.64 \\ \hline
        2 & MySQL      & Реляционная               & 1172.46 \\ \hline
        3 & MsSQL      & Реляционная               & 920.09  \\ \hline
        4 & PostgreSQL & Реляционная               & 617.90  \\ \hline
        5 & MongoDB    & Документо-ориентированная & 436.61  \\ \hline
        6 & Redis      & Ключ-значение             & 168.13  \\ \hline
    \end{tabular}
\end{table}


\section{Реляционные базы данных}

Идея реляционной модели была предложена Э. Ф. Коддом в 1970 году~\cite{Codd_Relational}.
В реляционной модели данные хранятся в виде таблиц связанных отношениями.
Отношения могут описаны логически в форме 
<<один-к-одному>> <<один-к-многим>> <<многие-к-многим>>.
Визуально представить такую модель можно с помощью нотации Чена, 
предложенной для описания моделей <<сущность-связь>> в 1976 году~\cite{Chen}.

Данные в реляционной модели должны удовлетворять следующим свойствам:
\begin{enumerate}
    \item каждая строка таблицы отображает кортеж или запись;
    \item все значения атрибутов атомарны;
    \item все строки уникальны;
    \item порядок строк не значим;
    \item порядок колонок не значим.
\end{enumerate}

\clearpage

\subsection{Основные понятия}
Рассмотрим основные понятия реляционной модели данных.


\subsubsection{Ключ}

Понятие ключа является одним из основных для РСУБД.
Ключ позволяет идентифицировать запись и облегчает создание отношения между сущностями.
Засчет ключей достигается важное свойство реляционной модели -- уникальность записей.

Есть три основных типа ключей:  
\begin{itemize}[label=---]
  
    \item \textbf{Потенциальный ключ} используется для того,
    чтобы определить сущность без обращения к другим данным таблицы.
    Представляет собой минимальный набор атрибутов для идентификации сущности.
    Потенциальных ключей может быть несколько.

    \item \textbf{Первичный ключ} используется для идентификации записи в таблице.
    Первичный ключ может быть только один и выбирается из множества потенциальных ключей.
    Засчет первичного ключа обеспечивается уникальность и отсутствие избыточности данных.
    На рисунке \ref{img:diploma-FK} поля <<ID>> сущностей <<человек>> и <<собака>> 
    являются первичными ключами.

    \item \textbf{Внешний ключ} ссылается на соответствующий первичный ключ другой сущности для того,
    чтобы однозначно определить эту другую сущность.
    На рисунке \ref{img:diploma-FK} поле <<хозяин>> является внешним ключом сущности <<собака>> и ссылается на сущность <<человек>>
    определяя связь <<один-к-многим>> между человеком и собакой.

\end{itemize}

\clearpage

\includeimage{diploma-FK} % Имя файла без расширения (файл должен быть расположен в директории inc/img/)
    {f} % Обтекание (с обтеканием)
    {h} % Положение рисунка (см. wrapfigure из пакета wrapfig)
    {0.8\textwidth} % Ширина рисунка
{Отношение с использованием первичного и внешнего ключа} % Подпись рисунка


\subsubsection{Нормализация данных}

Нормализация -- это набор правил, согласно которым необходимо 
проектировать сущности или таблицы в реляционной модели для определения отношений через
первичные и внешние ключи. Эти правила нужны для решения проблемы сложных доменов
описанной Э. Ф. Коддом~\cite{Codd_Relational}.
Нормализация представляет собой процесс разбиения таблиц на подтаблицы облегчая
дальнейшее взаимодействие с данными в базе.

Можно выделить следующие преимущества нормализации:
\begin{itemize}[label=---]
    \item yстранение аномалий при обновлении;
    \item оптимизация запросов;
    \item обеспечение целостности данных;  
    \item гибкость и производительность при аггрегации, сортировках.
\end{itemize}

В 1979 Кодд ввел три типа нормализации, которые были названы нормальными формами~\cite{Codd_NF}:
\begin{itemize}[label=---]

  \item Первая нормальная форма (1NF) -- согласно первой нормальной форме 
    не должно присутствовать повторяющихся групп.
    Если различные колонки таблицы содержат одинаковый тип информации,
    тогда таблица не соответствует 1NF.
    Данные должны быть разбиты на минимально допустимые фрагменты.
    Согласно Кодду все данные в таблице должны быть неделимы.
    
    \item Вторая нормальная форма (2NF) -- вторая нормальная форма наследует требования 1NF.
    В 2NF все неключевые атрибуты должны быть функционально зависимыми от первичного ключа.
    
    \item Третья нормальная форма (3NF) -- третья нормальная форма удовлетворяет всем требованиям 1NF и 2NF.
    В 3NF необходимо, чтобы все неключевые атрибуты должны зависеть от первичного ключа и не зависеть от неключевых атрибутов.

  \end{itemize}


\subsection{Язык запросов}

Реляционные базы данных используют сильно-структурированный язык для доступа к базе и последующего получения данных.
Этот язык получил название <<Structured Query Language>> (SQL).
Данный язык был разработан Дональдом Чемберлейном и Реймондом Бойсом в 70е для получения доступа к реляционным базам данных~\cite{SQL}.

SQL быстро заработал популярность и в дальнейшем большинство разработчиков РСУБД внедрили его в свои продукты. 
Основной причиной популярности языка стало то, что он значительно облегчал разработчикам взаимодействие с базами данных,
что в свою очередь сокращало время на разработку и стоимость программного продукта в целом.
В следствие популярности Американский Национальный Институт Стандартизации (ANSI) разработал стандарт SQL -- ANSI SQL.
Несмотря на стандарт языка, множество СУБД используют собственные модификации языка SQL.


% На таблице \ref{table:sql} приведено перечислены используемые в различных популярных РСУБД диалекты языка SQL.
% \begin{table}[h]
%     \centering
%     \caption{Сравнительная таблица используемых в РСУБД диалектов SQL}
%     \label{table:sql}
%     \begin{tabular}{ |m{10em} |m{10em}| } 
%      \hline
%      \textbf{РСУБД} & \textbf{Диалект SQL} \\ \hline 
%      IBM DB2 & SQL \\  \hline
%      Oracle & PL/SQL \\ \hline
%      MsSQL Server & Transact SQL \\ \hline 
%      MySQL & SQL \\ \hline
%      PostgreSQL & PL/pgSQL \\ 
%      \hline
%     \end{tabular}
% \end{table}

\clearpage

\subsection{ACID Свойства}
Важный набор понятий, гарантирующий безопасность и надежность транзакции 
известен как ACID свойства~\cite{Date}.
Логическая единица работы которая производится внутри последовательности операций
называется транзакцией.
Для того, чтобы обеспечить безопасную, надёжную и согласованную транзакцию,
логическая единица работы должны удовлетворять четырем свойствам: 
неразрывность, правильность, изоляция и устойчивость. 
Аббревиатура ACID соответственно сформирована из первых букв слов 
Atomicity, Consistency, Isolation, Durability.

\begin{itemize}[label=---]
    \item Неразрывность (англ. Atomicity) -- ``Транзакции неразрывны (Всё или ничего)''.
    В транзакции либо выполняются все необходимые операции, либо изменения не производятся.

    \item Правильность (англ. Consistency) -- ``Транзакции переводят базу из одного правильного состояния в другое''.
    При этом правильность не обязана обеспечиваться на всех этапах транзакции. 
    
    \item Изолированность (англ. Isolation) -- ``Транзакции изолированы друг от друга''. 
    Промежуточное состояние транзакции не может быть получено любой другой транзакцией.
    Изоляция необходима для того чтобы поддерживать согласованность между транзакциями.
    Изменения производимые параллельным выполнением транзакций также должны быть изолированы друг от друга.

    \item Устойчивость (англ. Durability) -- После завершения успешной транзакции изменения должны быть зафиксированы в системе
    даже в случае выхода системы из строя.

\end{itemize}

\clearpage

\section{Нереляционные базы данных}
Нереляционные базы данных принято называть NoSQL от <<Not Only SQL>>,
так как они разрабатывались для распределенного хранения данных
и тем самым они выходят за рамки ограничений накладываемых реляционной моделью данных,
для которой был разработан язык SQL.

NoSQL базы данных проектируются с учетом потребностей в распределенных
и высоконагруженных облачных вычислениях на больших объемах данных.
Также зачастую к ним предъявляется требование иметь гибкую схему 
для эффективного хранения неструктурированных данных~\cite{nosqlusage}.  

Можно выделить такие ключевые особенности NoSQL баз данных:
\begin{enumerate}
    \item Возможность горизонтального масштабирования.
    \item Возможность партиционирования и распределения по множеству серверов.
    \item Сравнительно ненадёжный параллелизм в сравнении с ACID.
    \item Схожий с SQL синтаксис. Простой язык запросов.
    \item Возможность динамически менять количество атрибутов.
    \item Эффективное использование ОЗУ и распределенных индексов для хранения данных.
\end{enumerate}

Горизонтальное масштабирование, репликация и распределение данных по нескольким серверам
дают возможность более быстро осуществлять операции чтения и записи данных.
Однако, NoSQL не удовлетворяет свойствам ACID,
которые обеспечивают правильность параллельных транзакций.
{Веб-приложения} в основном выполняются в распределенной среде,
где основным требованием является возможность к масштабированию
и согласно~<<CAP-теореме>> для распределенной системы не предоставляется
возможным обеспечить одновременно
правильность, доступность и устойчивость к разделению. 
Одновременно выполняться могут только лишь два свойства из трех.

%Eще цитатка
Менее строгая модель BASE (Basically Avaliable, Soft state, Eventual consistency) заменяет ACID в рамках NoSQL~\cite{nosql}.
\begin{itemize}[label=---]
    \item Базовая доступность (англ. Base avaliability) -- На любой запрос будет получен ответ о успешном или неуспешном завершении.
    \item Неустойчивое состояние (англ. Soft state) -- Состояние системы неустойчиво -- может изменяться со временем. 
    Причем изменения могут происходить без внешнего взаимодействия. 
    \item Согласованность в конечном счёте (англ. Eventual consistency) -- В конечном счете база данных будет правильной и согласованной, несмотря на то,
    что она может быть несогласованна в момент времени.
\end{itemize}

Структуры данных в нереляционных базах данных отличаются в сравнении с используемыми в реляционной модели.
Далее будут описаны некоторые типы моделей данных используемые в NoSQL базах данных.
\clearpage
\subsection{Нереляционные модели данных}
В данном разделе будут рассмотрены разные типы нереляционных моделей данных.
Согласно подходу к хранению данных нереляционные базы данных можно классифицировать по перечисленным моделям:
\begin{itemize}[label=---]
    \item Столбцовая (Колоночная) модель -- Несмотря на то, что и колоночная и реляционная база основываются на понятии строк и колонок,
    для колоночного хранилища нет необходимости определять колонки. % 19 цитата
    В сравнении с РСУБД, где данные хранятся в виде строк в таблице, колоночное хранилище предполагает хранение данных как набор колонок.
    Колонка может включать себя множество атрибутов, поэтому её могут называть <<Широкой>>.
    Это позволяет оптимизировать выполнение сложных запросов.
    Основные причины использования колоночного хранилища:
    \begin{itemize}[label=---]
        \item в качестве распределенного хранилища;
        \item для пакетной обработки больших объемов данных,
        предполагающих операции аггрегации,
        сортировки либо другой обработки данных;
        \item для создания аналитических хранилищ. 
    \end{itemize} 
    
    Множество популярных NoSQL СУБД используют колоночную модель данных, например: 
    Apache Cassandra, HBase, Bigtable, DynamoDB.

    \item Документо-ориентированная модель -- хранилище спроектированное для хранения данных в виде документов,
    которое позволяют записывать, получить и изменять слабо-структурированные данные.
    Для того, чтобы облегчить работу разработчика документы в документо-ориентированном хранилище не зависят от конкретной схемы.
    В документо-ориентированной базе данных, все данные представляются в виде документа в определенном формате.
    Документы могут быть в таких форматах как: 
    Некоторые форматы документов: XML, JSON, BSON, YAML и другие.
    На рисунке \ref{lst:example.json} представлен документ в формате JSON.
    Также следует отметить, что документ может содержать другой документ внутри себя.

    \clearpage

    \includelisting{example.json}{Пример структуры данных в документо-ориентированной БД}


    В документ-ориентированных СУБД каждый документ обязан иметь уникальный идентификатор за его генерацию зачастую отвечает СУБД.
    В основном документ-ориентированная модель используется для создания web-приложений, имеющих потребность в обработке крупных объемов распределенных по сети текстовых данных.
    В качестве примеров популярных документо-ориентированных СУБД можно привести:
    MongoDB, CouchDB, Terrastore.

    \item Модель Ключ-Значение -- в модели ключ-значение отсутствует схема и все данные хранятся в единой хэш-таблице.
    Идентификаторы или ключи символьные и могут быть как сгенерированными системой, так и заданными разработчиком вручную, аналогично идентификатору документа в документо-ориентированной базе данных.
    Хранилища типа ключ-значение в основном используются в виде `in-memory' распределенного кэша, для организации быстрого доступа к данным.
    Среди наиболее популярных СУБД основанных на модели ключ-значение есть такие как Memcached, MemcacheDB и Redis.

    \item Графовая модель -- Графовые базы данных хранят данные в виде графовых структур, состоящих из узлов, рёбер и свойств.
    Узлы отображают концептуальные объекты, соединенные линиями, которые называют ребрами. Ребра используются для создания связей между узлами и свойствами.
    Как в реляционной модели, графовые базы данных работают с отношениями путем прохождения по ребрам.
    Используя графовые алгоритмы графовые базы данных могут хранить данные в таком виде, который можно масштабировать по нескольким серверам для дальнейших связей ребрами.
    Узлы и отношения -- базовые понятия графовой модели, в которой узлы организуются по свойствам ассоциированным с отношениями, где данные хранимые в узлах также имеют собственные свойства.

    Графовые базы данных обычно используются в случаях, когда отношения между данными более важны. Например, многие социальные сети используют графовые базы данных.
    Среди известных графовых СУБД можно выделить Neo4J, FlockDB, AllegroGraph.
\end{itemize}

% \clearpage
 
\subsection{Теорема Брюера (CAP)}
В 2000г. Эриком Брюером была выдвинута CAP-теорема~\cite{cap,cap_new}. 
Основная ее идея звучит как: ``В любой реализации параллельных вычислений нельзя добиться одновременно не более двух свойств, таких как согласованность данных, доступность и устойчивость к разделению''.
Каждой системе необходимо соответствовать следующим свойствам~\cite{morecap}:
\begin{itemize}[label=---]
    \item согласованность данных (англ. Consistency) -- данные в каждом угле не должны противоречить друг другу и имеют общее согласованное состояние;
    \item доступность (англ. Avaliability) -- любой запрос к распределенной системе завершается корректно;
    \item устойчивость к разделению (англ. Partition tolerance) -- расщепление распределенной системы на изолированные сегменты не приводит к некорректности работы системы. Таким образом при отказе одного из узлов гарантируется работа общей системы.
\end{itemize}

Однако удовлетворить всем трем свойствам не предоставляется возможным.
Согласованность данных легко достигается в реляционной базе данных, в силу соответствия ACID.
В то же время, РСУБД проблематично горизонтально масштабировать,
однако NoSQL базы данных просто масштабируются,
но не могут предоставить тот же уровень согласованности данных,
в силу соответствия BASE.

% \clearpage
\subsection{Сравнение NoSQL баз данных}
На таблице \ref{table:cap} указаны результаты сравнения NoSQL СУБД разных моделей данных.
\begin{table}[h!]
    \centering
    \caption{Сравнение наиболее популярных NoSQL СУБД.}
    \label{table:cap}
    \begin{adjustbox}{width=\textwidth, angle=0}
        \begin{tabular}{|l|l|l|l|l|}
            \hline
            Название &
              Модель данных &
              \begin{tabular}[c]{@{}l@{}}Выполняемые свойства\\ CAP-теоремы\end{tabular} &
              \begin{tabular}[c]{@{}l@{}}Модель \\ целостности\end{tabular} &
              \begin{tabular}[c]{@{}l@{}}Область \\ применения\end{tabular} \\ \hline
            Redis &
              Ключ-значение &
              \begin{tabular}[c]{@{}l@{}}Согласованность \\ Устойчивость к  разделению\end{tabular} &
              BASE &
              \begin{tabular}[c]{@{}l@{}}Кэширование данных\\  сессий\end{tabular} \\ \hline
            MongoDB &
              \begin{tabular}[c]{@{}l@{}}Документо-\\ ориентированная\end{tabular} &
              \begin{tabular}[c]{@{}l@{}}Согласованность\\ Устойчивость к разделению\end{tabular} &
              BASE &
              \begin{tabular}[c]{@{}l@{}}Хранение больших\\ коллекций данных\end{tabular} \\ \hline
            Cassandra &
              \begin{tabular}[c]{@{}l@{}}Колоночно-\\ ориентированная\end{tabular} &
              \begin{tabular}[c]{@{}l@{}}Доступность\\ Устойчивость к разделению\end{tabular} &
              BASE &
              \begin{tabular}[c]{@{}l@{}}Распределенное хранение \\ и обработка больших \\ данных\end{tabular} \\ \hline
            Neo4j &
              Графовая &
              \begin{tabular}[c]{@{}l@{}}Согласованность\\ Доступность\end{tabular} &
              ACID &
              \begin{tabular}[c]{@{}l@{}}Хранение и анализ\\ социальных графов\end{tabular} \\ \hline
            \end{tabular}
    \end{adjustbox}
\end{table}


\subsection*{Выбор нереляционной базы данных}

Для более детального рассмотрения был выбран процесс миграции данных с РСУБД на документо-ориентированную СУБД.
Выбор документо-ориентированной СУБД обусловлен тем, 
что документо-ориентированные СУБД являются вторыми по распространенности после реляционных~\cite{db_rating}.
Также стоит учесть, что принципы документо-ориентированной модели явно противоречат реляционной,
так как оптимальным подходом является денормализация схемы, в противовес нормализации.
Помимо этого открытыми являются проблемы избыточности данных после миграции 
и переноса отношений <<многие-к-многим>>.

\clearpage

\subsection{Сравнительный анализ реляционных и нереляционных баз данных}

\begin{itemize}[label=---]
    \item Надежность транзакции: РСУБД соответствует ACID и предоставляет надежность транзакции.
    Нереляционные базы, опираются на ослабленную спецификацию --- BASE.
    \item Модель данных: РСУБД основаны на реляционной модели, и данные представляются в виде таблиц с набором строк связанных отношением.
    Нереляционные базы данных используют различные модели, например: ключ-значение, документо-ориентированные, колоночные и графовые.
    \item Масштабируемость: Web-приложения требуют возможности горизонтального масштабирования, т.\,к. они распределены по нескольким серверам.
    NoSQL базы данных поддерживают горизонтальное масштабирование, в свою очередь горизонтальное масштабирование РСУБД является трудоемкой задачей.
    \item Обработка больших данных: Из за проблем с масштабированием и партицированием данных в распределенной кластеризованной среде, обработка больших объемов данных на РСУБД проблематична.
    \item Устойчивость к сбоям: Данные в РСУБД консистентны и поддерживаются в правильном состоянии за счет транзакций и журналирования.
    Сохранность данных в нереляционных базах зависит от правильности их репликации.
    \item Безопасность: В РСУБД внедряются сложные механизмы безопасности.
    Нереляционные базы спроектированы таким образом, чтобы добиться производительности в обработке больших объемов данных, засчет потерь в их безопасности.
    Безопасность данных является крупным объектом обсуждения в рамках активно развивающихся облачных систем.
\end{itemize}

\clearpage

В таблице \ref{table:sqlnosql} приведено сравнение реляционных и нереляционных СУБД.

\begin{table}[h]
    \centering
    \caption{Сравнение РСУБД и NoSQL СУБД}
    \label{table:sqlnosql}
    \begin{tabular}{|l|l|l|}
    \hline
    \backslashbox{Критерий}{Вид} &
      РСУБД &
      Нереляционные СУБД \\ \hline
      \begin{tabular}[c]{@{}l@{}}Надежность\\ транзакции\end{tabular} &
      Да (ACID) &
      Нет (BASE) \\ \hline
    Модель данных &
      Реляционная &
      \begin{tabular}[c]{@{}l@{}}Множество\\ нереляционных\end{tabular} \\ \hline
    \begin{tabular}[c]{@{}l@{}}Горизонтальное\\ масштабирование\end{tabular} &
      Нет &
      Да \\ \hline
    \begin{tabular}[c]{@{}l@{}}Обработка \\ больших данных\end{tabular} &
      \begin{tabular}[c]{@{}l@{}}Не предназначались,\\ имеют ограничения\end{tabular} &
      \begin{tabular}[c]{@{}l@{}}Учитывают потребности в\\обработке больших\\данных\end{tabular} \\ \hline
    Надежность &
      Да &
      Зависит от репликации \\ \hline
    Безопасность &
      Да & Нет\\ \hline
    \end{tabular}
\end{table}

\clearpage

\section{Методы миграции}

Миграция -- процесс переноса данных из одной или более текущих баз данных
в одну или более новых баз данных с возможным изменением структуры данных.

Промышленные применения реляционной модели данных не поддерживают растущие требования
по производительности в сравнении с нереляционными в рамках анализа больших объемов данных~\cite{migration}.
Проведя сравнительный анализ реляционных и нереляционных СУБД, можно сделать вывод, 
что основное различие состоит в структуре хранения данных.

Существуют различные подходы к миграции данных.
Они предлагают различную методологию по переносу данных и схемы 
из реляционного хранилища в документо-ориентированное.
Основной задачей, при переносе схемы и данных является представление реляционных отношений.
Методы могут быть основаны как на анализе и преобразовании схемы данных, 
так и не зависеть от схемы и манипулировать только данными.

\subsection{Описание процесса миграции}
С учетом структуры и деталей хранения, нереляционные СУБД отличаются от реляционных.
Реляционная модель строго структурирована и данные нормализованы в таблицах согласно их отношениям,
в нереляционных базах данные хранятся в полуструктурированном или неструктурированном виде 
и обычно денормализованы.
Таким образом процесс миграции не является тривиальной задачей.

На рис \ref{img:migration} представлена общая схема процесса миграции.
\includeimage
  {migration} % Имя файла без расширения (файл должен быть расположен в директории inc/img/)
  {f} % Обтекание (с обтеканием)
  {h!} % Положение рисунка (см. wrapfigure из пакета wrapfig)
  {0.7\textwidth} % Ширина рисунка
{Схема процесса миграции} % Подпись рисунка

Далее будут рассмотрены несколько методов миграции.

\clearpage
\subsection{Методы на основе аггрегации}
Одни из наиболее распространенных подходов при миграции в документо-ориентированное хранилище из РСУБД основаны на аггрегации.
В основе метода лежит идея о денормализации структуры, путем аггрегирования всех зависимых сущностей в единую главную сущность.
Таким образом из множества таблиц и отношений между ними образуется единый документ, где все зависимые документы являются вложенными внутрь основного документа.
Подобные подходы позволяют получать тесно связанные данные одним запросом к одной большой сущности,
таким образом получается избежать JOIN-операций, не поддерживаемых в множестве нереляционных СУБД,
однако миграция сложных схем подобным образом приводит к большой избыточности данных и их повторению.
Также стоит отметить, что в случае повторения данных необходимо самостоятельно следить за их согласованностью.

В статье \cite{embedding} представлен полу-автоматизированный метод преобразования реляционной базы данных в документо-ориентированную на основе табличной денормализации.
Пользователю предлагается выбрать необходимые для переноса главную и зависимые сущности. 
Далее все зависимые сущности переносятся в виде вложенных в главный документов.
Из недостатков этого метода можно выделить то, что он не позволяет сразу перенести всю реляционную структуру и требует вмешательства пользователя.
Также при излишней денормализации растет объем документа и как следствие выполнение сложных запросов с несколькими коллекциями становится менее эффективным.
Помимо этого такой подход неприменим к сложным реляционным структурам, так как конверсия в единый денормализованный документ не всегда возможна.

\clearpage

На рисунке \ref{img:tableembed} представлена общая схема алгоритма на основе табличной денормализации.
\includeimage
    {tableembed} % Имя файла без расширения (файл должен быть расположен в директории inc/img/)
    {f} % Обтекание (с обтеканием)
    {h!} % Положение рисунка (см. wrapfigure из пакета wrapfig)
    {0.5\textwidth} % Ширина рисунка
    {Алгоритм миграции на основе табличной денормализации} % Подпись рисунка

\clearpage

Также существуют полностью автоматизированные методы денормализации.
В статье \cite{graphmethod} представлен метод миграции схемы основанный на анализе графа отношений.
Предлагается рассматривать реляционную схему как направленный ациклический граф.
Далее на основе понятия полноты данных вводятся операции расширения узла.
Так как изначально каждый узел представляет собой сущность или таблицу в результате работы алгоритма граф преобразуется таким образом, что в каждом узле хранится вся зависимая от него информация.

На рисунке \ref{img:graph} приведена схема алгоритма для поиска цепочки всех узлов необходимых для дополнения сущности.
В качестве входных данных принимаются значения:
$$O[v] \leftarrow |b(v)|,$$
$$P \leftarrow  \{v | v \in V \wedge O[v] = 0\},$$
$$Q \leftarrow V - P,$$
$$T \leftarrow \emptyset,$$
$$ S \leftarrow \{\},$$
где:
\begin{itemize}[label=---]
    \item $O[v]$ -- количество входящих граней;
    \item $Q$ -- узлы у который $O[v] = 0$;
    \item $P$ -- узлы имеющие входящие грани и не имеющие исходящих;
    \item $T$ -- узлы не имеющие ни входящих не исходящих граней;
    \item $S$ -- путь по графу, составляющий полную сущность;
    \item $f(x)$ -- узлы, имеющие прямой путь в $x$.
\end{itemize}

\clearpage

\includeimage
    {graph} % Имя файла без расширения (файл должен быть расположен в директории inc/img/)
    {f} % Обтекание (с обтеканием)
    {h} % Положение рисунка (см. wrapfigure из пакета wrapfig)
    {0.9\textwidth} % Ширина рисунка
    {Алгоритм миграции на основе расширения узлов направленного ациклического графа} % Подпись рисунка
    
В итоге получаем несколько денормализованных коллекций документов, с которыми можно работать не прибегая к JOIN-операциям.
Однако такой подход ведет к высокой избыточности, так как все зависимые данные будут дублироваться.
Также к недостаткам метода можно отнести то, что не все графы отношений можно выразить с помощью направленного ациклического графа.
Таким образом подобный метод не подходит для переноса схем имеющих отношения <<многие-к-многим>>. 

\clearpage
\subsection{Методы на основе нормализации}
Несмотря на то, что документо-ориентированная схема предполагает денормализацию данных,
существуют методы напрямую переносящие нормализованную реляционную схему в документо-ориентированную модель.
В таком случае отношения реализуются не засчет вложенности документов, а с помощью внешних ключей.
Внешние ключи определяются так же как и в РСУБД -- через указание первичного ключа (идентификатора) другой сущности (документа).

В статье \cite{reference} описывается каким образом нормализованные данные из РСУБД могут быть представлены в документо-ориентированной СУБД без нарушения нормальной формы.
На основе этого может быть построен алгоритм согласно которому все данные из РСУБД переносятся с сохранением всех своих внешних и первичных ключей.

Однако, в силу того, что документо-ориентированные СУБД не подразумевают нормализованное представление данных, сохранение согласованности данных и реализация JOIN-операций отводится разработчику.
Это значительно затрудняет операции вставки, изменения и удаления, так как СУБД не берет на себя отслеживание изменений в зависимых сущностях. 

\clearpage
На рисунке \ref{img:refmodel} представлена схема алгоритма переноса нормализованной схемы.
\includeimage
    {refmodel} % Имя файла без расширения (файл должен быть расположен в директории inc/img/)
    {f} % Обтекание (с обтеканием)
    {h} % Положение рисунка (см. wrapfigure из пакета wrapfig)
    {0.7\textwidth} % Ширина рисунка
    {Алгоритм переноса нормализованной схемы с использованием ссылок} % Подпись рисунка

\clearpage
В статье \cite{csv} предлагается подход на основе перевода таблицы из РСУБД в формат CSV (Comma Separated Value) и далее загрузке данного файла в документо-ориентированную СУБД.
Данный метод очень сильно зависит от того, каким образом документо-ориентированная СУБД будет воспринимать полученный формат файла.
Зачастую при использовании подобного подхода могут не учитываться отношения реляционной модели и тем самым мы теряем возможность обращения к связанным между собой данным.
Поэтому данный подход предпочтителен, когда необходимо перенести конкретную таблицу из реляционной в документо-ориентированную базу данных.
На рисунке \ref{img:CSV} представлена общая схема метода миграции через CSV файл.
\includeimage
    {CSV} % Имя файла без расширения (файл должен быть расположен в директории inc/img/)
    {f} % Обтекание (с обтеканием)
    {h} % Положение рисунка (см. wrapfigure из пакета wrapfig)
    {0.5\textwidth} % Ширина рисунка
    {Алгоритм миграции на основе перевода в CSV файл} % Подпись рисунка

\clearpage

\subsection{Метод основанный на введении нереляционной схемы}
Методы вводящие понятие собственной нереляционной схемы частично объединяют описанные выше подходы.
Зачастую предлагается введение методологии или набора правил,
согласно которым реляционная схема приводится к виду совместимому с документо-ориентированной моделью.
В статье \cite{DODS} описывается один из таких методов.
Предлагается набор правил, согласно которому на основе вида сущности и отношения принимается решение о том, каким образом реализуется данное отношение в документо-ориентированной модели.
Отношение может быть реализовано как с помощью ссылки, так и зачет вложенности документов.



Метод предлагает подход для миграции данных в таком формате, который является компромиссным между нормализованным и денормализованным.
Также данный метод предлагает решение для переноса связей <<многие-к-многим>> хоть и не решает проблем с отслеживанием согласованности данных в рамках этого отношения. 
Засчет этого он может использоваться для переноса сложных реляционных структур, в то время как методы на основе аггрегации зачастую не применимы к структурам,
которые нельзя представить в виде направленного ациклического графа. 

Таким образом предлагается оптимальное решение не приводящее к излишней избыточности,
при этом более оптимальное с точки зрения хранения данных и выполнения запросов нежели перенос нормализованной структуры с сохранением нормальной формы.
\clearpage
На рисунке \ref{img:dods} представлена общая схема алгоритма преобразования реляционной схемы в документо-ориентированную согласно методологии описанной в \cite{DODS};
\includeimage
    {dods} % Имя файла без расширения (файл должен быть расположен в директории inc/img/)
    {f} % Обтекание (с обтеканием)
    {h} % Положение рисунка (см. wrapfigure из пакета wrapfig)
    {0.75\textwidth} % Ширина рисунка
    {Алгоритм миграции на основе нереляционной схемы DODS} % Подпись рисунка

\clearpage

\subsection{Метод использующий оберточный слой для запросов}
Методы использующие оберточный слой для запросов позволяют проводить процесс миграции без анализа схемы исходной реляционной базы данных.
Сравнение конкретных систем использующих этот метод представлено в статье~\cite{mappers}.
Подход с оберточным слоем предполагает создание слоя преобразующего запросы на языке SQL в запросы совместимые с желаемой нереляционной СУБД.
Таким образом данные миграция данных осуществляется засчет средств языка SQL, который поддерживается большинством реляционных СУБД.
Такой подход позволяет объединить преимущества хранения в документо-ориентированной модели и возможности языка запросов SQL,
однако результирующая структура данных может быть не оптимальной в рамках нереляционной модели данных.

На рисунке \ref{img:MAPPERS} представлена общая схема метода использующего оберточный слой для запросов.

\includeimage
  {MAPPERS} % Имя файла без расширения (файл должен быть расположен в директории inc/img/)
  {f} % Обтекание (с обтеканием)
  {h!} % Положение рисунка (см. wrapfigure из пакета wrapfig)
  {0.2\textwidth} % Ширина рисунка
{Метод миграции использующий оберточный слой для запросов} % Подпись рисунка

\clearpage

\subsection{Методы на основе объектного преобразования}
Существуют методы на основе объектных преобразований, они используют объектно-реляционные (ORM) и объектно-нереляционные (ONM) преобразования.
Использование таких систем позволяет ввести промежуточное представление сущности в виде объекта,
что помогает создать единый интерфейс взаимодействия с базами данных независимо от модели,
а также облегчает разработчикам взаимодействие с данными в рамках объектно-ориентированной парадигмы.
В работе~\cite{ONM} рассмотрено несколько подобных систем.

Например, Hibernate широко используется для работы и с реляционными и нереляционными базами данных и предлагает собственные языки запросов HQL и JPQL,
что позволяет единообразно обращаться к многим видам баз данных~\cite{hibernateogm}.
Такие системы получили популярность так как облегчают наиболее распространенные операции получения и изменения данных,
однако они не являются оптимальными для работы в больших приложениях,
где требуется множество сложных аналитических запросов к данным.
Аналогичный подход с использованием системы LinQ рассмотрен в статье~\cite{linq}.

На рисунке \ref{img:ORM} представлена общая схема метода миграции на основе объектного преобразования.
\includeimage
    {ORM} % Имя файла без расширения (файл должен быть расположен в директории inc/img/)
    {f} % Обтекание (с обтеканием)
    {h!} % Положение рисунка (см. wrapfigure из пакета wrapfig)
    {0.8\textwidth} % Ширина рисунка
{Алгоритм миграции на основе объектного преобразования} % Подпись рисунка

\clearpage

\subsection{Методы на основе ETL}
ETL (Extract Transform Load) -- подход к обработке данных, предполагающий последовательное получение данных, преобразование и выгрузку.
На основе этого подхода основаны некоторые методы миграции данных.
В статье~\cite{etl} предложен метод миграции данных из реляционной БД в документо-ориентированную основанный на ETL подходе.
Сначала путём подключения к реляционной БД получается структура таблиц.
Далее пользователю предоставляется возможность выбрать данные необходимые для переноса в новую БД.
После выбора необходимых таблиц и аттрибутов, происходит выгрузка данных и преобразование их в формат совместимый с документо-ориентированной моделью, затем данные загружаются в новую базу данных. 
Метод нельзя назвать полностью автоматизированным, так как этап определения новой схемы требует вмешательства пользователя, однако этот метод автоматизирует этап переноса данных в выбранную пользователем схему.

Cуществуют решения с открытым исходным кодом использующим данный подход, например: MongoSyphon~\cite{mongosyphon}.

% \clearpage
На рисунке \ref{img:ETL} представлена общая схема ETL подхода.
\includeimage
    {ETL} % Имя файла без расширения (файл должен быть расположен в директории inc/img/)
    {f} % Обтекание (с обтеканием)
    {h!} % Положение рисунка (см. wrapfigure из пакета wrapfig)
    {0.55\textwidth} % Ширина рисунка
    {Алгоритм миграции на основе ETL} % Подпись рисунка

\clearpage

\section{Сравнение методов миграции}
В ходе исследования были установлены следующие критерии сравнения методов миграции:
\begin{itemize}[label=---]
    \item применимость -- применимость метода включает в себя два критерия: применимость к схемам и применимость к сущностям.
    
    Применимость к схемам показывает, к каким схемам наиболее применим данный метод.
    Схема называется сложной если она включает в себя более 10 сущностей.

    Применимость к сущностям показывает, к миграции сущностей какого размера применим данный метод.
    Сущность называется большой если она насчитывает более 10 атрибутов и имеет более двух отношений.
    \item поддержка отношения <<многие-к-многим>> -- данный критерий показывает,
    позволяет ли метод переносить отношения типа <<многие-к-многим>> из РСУБД в документо-ориентированную;
    \item избыточность данных -- данный критерий определяет количество дублирующейся информации при миграции.
    Избыточность определяется как высокая, если все зависимые данные дублируются для каждой сущности или как низкая, если дублирование данных не присутствует.
    Средняя избыточность показывает, что данные дублируются, но в количестве меньшем количества зависимых сущностей;
    \item автономность -- данный критерий показывает потребность вмешательства разработчика в процесс миграции.
\end{itemize}

\clearpage
 
В таблице \ref{table:methodcomp} приведено сравнение методов миграции по сформулированным критериям.

\begin{table}[h!]
    \centering
    \caption{Сравнение методов миграции}
    \label{table:methodcomp}
    \begin{adjustbox}{width=\textwidth, angle=0}
        \begin{tabular}{|l|ll|l|l|l|}
            \hline
            \multirow{2}{*}{\backslashbox{Название}{Критерий}} &
              \multicolumn{2}{l|}{Применимость} &
              \multirow{2}{*}{Избыточность данных} &
              \multirow{2}{*}{\begin{tabular}[c]{@{}l@{}}Поддержка отношения\\ <<многие-к-многим>>\end{tabular}} &
              \multirow{2}{*}{Автономность} \\ \cline{2-3}
                                                                                                   & \multicolumn{1}{l|}{К схеме} & К сущностям &              &     &     \\ \hline
            \begin{tabular}[c]{@{}l@{}}Метод на основе \\ табличной \\ денормализации\end{tabular} & \multicolumn{1}{l|}{Простые} & Малые       & Максимальная & Нет & Нет \\ \hline
            \begin{tabular}[c]{@{}l@{}}Графовый метод \\ на основе аггрегации\end{tabular}         & \multicolumn{1}{l|}{Сложные} & Малые       & Максимальная & Нет & Да  \\ \hline
            \begin{tabular}[c]{@{}l@{}}Метод на основе\\ нормализации\end{tabular}                 & \multicolumn{1}{l|}{Сложные} & Большие     & Минимальная  & Да  & Да  \\ \hline
            \begin{tabular}[c]{@{}l@{}}Метод на основе\\ нереляционной \\ схемы\end{tabular}       & \multicolumn{1}{l|}{Любые}   & Любые       & Средняя      & Да  & Да  \\ \hline
            \begin{tabular}[c]{@{}l@{}}Метод на основе \\ обертки запросов \\ к СУБД\end{tabular} &
              \multicolumn{1}{l|}{Любые} &
              Любые &
              \begin{tabular}[c]{@{}l@{}}Средняя \\ (Зависит от реализации)\end{tabular} &
              Да &
              Нет \\ \hline
            \begin{tabular}[c]{@{}l@{}}Метод на основе\\ объектного\\ преобразования\end{tabular}  & \multicolumn{1}{l|}{Простые} & Малые       & Средняя      & Да  & Нет \\ \hline
            \begin{tabular}[c]{@{}l@{}}Метод на основе \\ ETL процесса\end{tabular} &
              \multicolumn{1}{l|}{Любые} &
              Большие &
              \begin{tabular}[c]{@{}l@{}}Средняя\\ (Зависит от реализации)\end{tabular} &
              Да &
              Нет \\ \hline
            \end{tabular}
\end{adjustbox}
\end{table}

\subsection*{Вывод}
Сравнение показало, что метод на основе введения нереляционной схемы является оптимальным, так как он имеет возможность полной автоматизации и предполагает возможность оптимизации количества дубликатов данных.
Также методология предлагаемая данным методом имеет возможность быть модифицированной на основе дополнительных входных данных.

\clearpage

\section{Применение миграции данных}
Задача миграции данных из реляционного в документо-ориентированное хранилище возникает по разным причинам:
\begin{enumerate}
    \item Масштабирование приложения --- при росте Web-приложений возникает потребность в увелечении объема и производительности хранилища,
    зачастую это достигается увеличением количества серверов и вынесением приложения в распределенную среду.
    Таким образом возникает потребность в переносе данных в более приспособленное для таких условий хранилище.
    
    \item Отказ от строгой структуры данных --- иногда оказывается, что спроектированная в ходе разработки реляционная схема не подходит для решаемой задачи.
    В таком случае возникает потребность в переносе данных в хранилище с менее строгой схемой, например документо-ориентированное.

    \item Аналитика --- накопление больших объемов данных затрудняет работу с РСУБД.
    В таких случаях прибегают к созданию нереляционных хранилищ, на которых проще и эффективнее проводить аналитику.
\end{enumerate}

\clearpage

\section{Постановка задачи миграции из реляционного в \mbox{документо-ориентированное} хранилище}
Задача миграции данных состоит в том, 
чтобы перенести данные из одной базы данных в другую наиболее оптимальным образом.
В силу того, что реляционная и документо-ориентированная модели
значительно отличаются такой процесс не является тривиальным.
Возникают проблемы в переносе отношений и организации доступа к данным в новой модели.
В ходе исследования было замечено, что большая часть методов не
учитывает особенности взаимодействия с ранее хранимыми данными.
Как следствие, в качестве метода решения задачи был выбран метод на основе введения нереляционной схемы.
Однако, рассмотренные в ходе работы методы никак не учитывают статистику запросов к РСУБД. 
Таким образом предлагается оптимизация процесса миграции на основе статистики запросов.
Например, в ходе миграции стоит учесть, что данные которые чаще запрашиваются совместно, стоит аггрегировать.

\clearpage

На рисунке \ref{img:diploma-A0} представлена постановка задачи в форме IDEF0-диаграммы.
\includeimage{diploma-A0}
    {f}
    {h}
    {\textwidth}
    {IDEF0-диаграмма метода миграции}
