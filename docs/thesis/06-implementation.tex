\chapter{Технологический раздел}
В данном разделе производится выбор средств программной реализации
метода миграции данных из реляционной в документо-ориентированную базу данных,
описывается формат входных и выходных данных.
Приводятся детали реализации программных компонентов, 
проводится проверка корректности работы разработанного метода, 
описывается взаимодействие пользователя с интерфейсом ПО, реализующим метод.

\section{Выбор средств реализации} 
Для реализации разрабатываемого метода был выбран язык Java версии 17~\cite{openjdk-17}.
Выбор языка обусловлен наличием опыта разработки и нативной поддержкой протокола JDBC,
что позволяет взаимодействие с большинством РСУБД посредством JDBC-драйвера~\cite{jdbc-api}.

В разработанном ПО были использованы следующие библиотеки и фреймворки:
\begin{itemize}[label=---]
    \item Spring Framework (Spring Boot)~\cite{spring-boot} --- фреймворк реализующий контейнер инверсии зависимостей, 
    используется для связывания модулей приложения и конфигурации;
    \item Spring Data JDBC~\cite{spring-data-jdbc} --- модуль интегрирующий JDBC в Spring Boot и
    предоставляющий верхнеуровневый интерфейс для обращения к базе данных;
    \item Spring Data MongoDB~\cite{spring-data-mongodb} --- модуль предоставляющий интерфейс обёртку для драйвера MongoDB;
    \item Spring Shell~\cite{spring-shell} --- модуль предоставляющий функциональность для разработки консольных интерфейсов;
    \item Jackson~\cite{jackson-databind} --- библиотека предоставляющая интерфейс для работы с JSON-объектами;
    \item pg\_stat\_statements~\cite{pg-stat-statements} --- модуль PostgreSQL логирующий запросы к СУБД.
\end{itemize}
 
\clearpage 


\clearpage
\section{Структура разработанного ПО}

Структура разработанного ПО представлена на рисунке \ref{img:component-diagram}.

\includeimage{component-diagram}{f}{h}{\textwidth}{Диаграмма компонентов}


По разделено на следующие основные компоненты:
\begin{itemize}[label=---]
    \item Компонент доступа к данным, отвечает за получение метаданных и выполнение запросов к БД;
    \item Компонент преобразования схемы, отвечает за формирование промежуточной схемы;
    \item Компонент миграции, производит миграцию данных согласно промежуточной схеме;
    \item Компонент пользовательского интерфейса, отвечает за взаимодействия пользователя с системой и ввод данных.
\end{itemize}

\clearpage

\subsection{Доступ к данным}

ПО содержит два компонента доступа к данным.
\begin{enumerate}
    \item Компонент доступа к PostgreSQL.
    \item Компонент доступа к MongoDB.
\end{enumerate}

Компонент доступа к MongoDB используется для записи данных в документо-ориентированную базу.

Компонент доступа к PostgreSQL используется для чтения, получения метаданных и истории запросов.

\subsubsection{Получение метаданных}
Получение метаданных о таблицах происходит по протоколу JDBC. 
Реализация протокола содержит класс \textmd{DatabaseMetadata} и метод \textmd{Connection::getMetadata}.
\textmd{DatabaseMetadata} содержит информацию о доступных схемах, таблицах, столбцах, ограничениях и ключах.

Каждая СУБД содержит собственные средства для получения метаданных. 
PostgreSQL реализует как \textmd{information\_schema}, описанную в стандарте SQL-92~\cite{sql-92},
так и собственный набор представлений \textmd{pg\_catalog}~\cite{pg-catalog}.

Перечисленные инструменты используются для получения следующей информации:
\begin{itemize}[label=---]
    \item список таблиц в схеме;
    \item метаданные столбцов (названия, типы данных);
    \item метаданные ключей (первичные и внешние ключи таблицы);
    \item метаданные отношений (внешние ключи, ссылающиеся на рассматриваемую таблицу).
\end{itemize}

Полученные метаданные позволяют сформировать граф, описывающий отношения между таблицами в схеме.
На листинге \ref{lst:GetTableMetadata.java} представлен метод получения метаданных о таблице.

\clearpage

\includelisting{GetTableMetadata.java}{Метода получения метаданных о таблице}

\clearpage

\subsubsection{Получение истории запросов}

Для получения истории запросов используется модуль \textmd{pg\_stat\_statements}. 
Этот модуль отвечает за логирование запросов к СУБД и подсчитывает количество выполненных запросов.

Путём аггрегирования записей в этой таблице можно получить информацию о том, 
как часто производятся \textmd{SELECT} операции над определенной таблицей.
Аналогично, возможно подсчитать количество \textmd{JOIN} операций между двумя таблицами.

Получив эти данные, можно выcчитать относительное количество JOIN-ов и 
сделать выбор в пользу того или иного способа реализации отношения в документо-ориентированной схеме.

На листинге \ref{lst:QueryLogRepository.java} представлены методы получения количества запросов.
\includelisting{QueryLogRepository.java}{Метода получения метаданных о таблице}

\clearpage

\subsection{Преобразование схемы}

Компонент преобразования схемы отвечает за формирование промежуточной 
документо-ориентированной схемы из полученных метаданных. 
На основе метаданных, информации о предметной области и кратности отношений формируется JSON-документ,
на основе которого будут преобразованны данные из реляционной базы данных.

Преобразование схемы происходит следующим образом:
\begin{enumerate}
    \item Производится выбор таблиц, которые будут отображены в коллекции документов.
    \item Начиная с каждой таблицы производится обход графа отношений в глубину,
    на каждом шаге которого производится либо перенос отношения в виде вложенного документа,
    либо в виде ссылки. Если отношение избыточно, его можно разорвать.
    \item Запрашивается информация о кратности отношения на уровне записи в таблице,
    в случае, если переносится отношение <<один-к-многим>> формируется массив.
    \item Метаданные используются для сохранения информации о отображении из РСУБД.
    \item Формируется промежуточная схема, содержащая информацию о том, 
    какие данные могут быть перенесены напрямую, а какие требуют дополнительных запросов.
    Для этого созданные ссылки и вложенные документы содержат информацию о источнике.
\end{enumerate}

\clearpage

На листинге \ref{lst:clientSchema.json} представлены методы получения количества запросов.
\includelisting{clientSchema.json}{Листинг промежуточной схемы}

\clearpage

\subsection{Миграция данных}
Компонент миграции отвечает за перенос данных согласно сформированной схеме.
Перед началом процесса миграции формируется очередь по наличию ссылок, 
таким образом в процессе миграции не произойдет попытки создать ссылку на документ, 
который еще не был создан.

Далее по очереди происходит формирование всех необходимых коллекций.

Сначала формируется представление документа верхнего уровня, далее параллельно для каждого документа, 
рекурсивно происходит формирование всех вложенных документов. 
Отдельно формируются запросы на формирование ссылок.

В результате миграции данных в MongoDB появляется отображение требуемых данных из PostgreSQL.

На листинге \ref{lst:queue.java} представлен код формирования очереди миграции.
\includelisting{queue.java}{Формирование очереди}

\clearpage

На листинге \ref{lst:migrateSchema.java} представлен код миграции данных согласно сформированной схеме.
\includelisting{migrateSchema.java}{Миграция данных по схеме}

\clearpage
\section{Проверка корректности работы ПО}
Тестирование разработанного ПО проводилось несколькими путями. 
При разработке использовались модульные тесты, 
которые определяли работу модулей преобразования схемы, правильность переноса промежуточной схемы и отображение типов.

Помимо этого проводилось тестирование на нескольких базах данных.

На рисунке \ref{img:test-er} представлена ER-диаграмма одной из баз на которой проводилось тестирование.

\includeimage{test-er}{f}{h}{\textwidth}{ER-диаграмма тестовой базы данных}

\clearpage

В ходе тестирования производилась миграция базы данных, 
после чего производилась проверка на наличие эквивалентного отображения каждой записи из 
реляционной базы данных в документо-ориентированной.

Каждый созданный документ верхнего уровня хранит копию первичного ключа из РСУБД,
что позволяет проверить каждая ли запись была перенесена путём выполнения запросов 
с поиском по одному значению в обе базы данных.

Также проверялось наличие отношений, которые были отброшены, 
так как в результирующей базе не должно присутствовать отношений, 
которые были разорваны в процессе миграции.

На листинге \ref{lst:client.json} представлен пример результирующего документа в MongoDB.
\includelisting{client.json}{Документ коллекции clients}


\clearpage


\section{Пользовательский интерфейс}
Разработанное ПО поставляется в виде файла с расширением \textmd{.jar} и конфигурационного файла.

Запуск программы происходит командой \textmd{java -jar MigrationTool.jar}. 
В одной директории с программой должен находиться конфигурационный файл \textmd{application.properties}.
На листинге \ref{lst:configuration.properties} представлена структура конфигурационного файла.
\includelisting{configuration.properties}{Листинг файла конфигурации}

В конфигурационном файле указываются параметры подключения к реляционной и документо-ориентированной СУБД:
адрес, порт, название базы, параметры аутентификации.

Пользовательский интерфейс представляет собой консольное приложение, реализованное с использованием Spring Shell.
При запуске приложения пользователь попадает в интерактивную среду принимающую команды для исполнения.
Список всех доступных команд может быть получен командой \textmd{help}.

На рисунке \ref{img:ui-help} представлен результат выполнения команды \textmd{help}.
\includeimage{ui-help}{f}{h}{\textwidth}{Список доступных команд}

Для того чтобы приступить к процессу миграции необходимо ввести команду \textbf{migration init}. 
После выполнения команды пользователь попадает в меню, 
в котором можно ввести схему в которой будет происходить работа
и произвести выбор таблиц на основе которых будут созданы коллекции документов.

\clearpage

На рисунке \ref{img:ui-init} представлено меню выбора таблиц.
\includeimage{ui-init}{f}{h}{\textwidth}{Меню выбора таблиц}

Во время выбора таблиц допускается фильтрация по имени таблицы путём ввода с клавиатуры.
Выбор осуществляется засчет стрелок и нажатия клавиши <<Пробел>>, 
при нажатии на клавишу <<Ввод>> выполнение команды завершится.

Список выбранных таблиц, а также введенная схема будут сохранены.
После чего будет выполнен сбор метаданных и возврат в исходную интерактивную среду.

Для того чтобы продолжить процесс миграции следует ввести команду \textbf{migration start}.
При вводе этой команды пользователь попадает в меню миграции таблиц, 
в котором необходимо по очереди провести миграцию каждой таблицы, 
пока не останется не мигрированных таблиц.

Перечислены таблицы, их статус, а также количество внешних ключей и внешних отношений с другими таблицами.

На рисунке \ref{img:ui-start} представлено меню выбора таблиц для миграции.
\includeimage{ui-start}{f}{h}{\textwidth}{Меню миграции таблиц}

\clearpage

При выборе таблиц, происходит переход в режим миграции таблицы.
В нем на экран выводится рассматриваемая таблица и перечисляются отношения которые необходимо мигрировать.
Пользователь находится в режиме миграции таблицы, пока не будут мигрированы все отношения таблицы.

На рисунке \ref{img:ui-table} представлено меню выбора отношения для миграции.
\includeimage{ui-table}{f}{h}{\textwidth}{Меню миграции таблицы}

\clearpage

Далее при выборе отношения необходимо выбрать каким образом будет мигрировано отношение.
На данном этапе выводится информация о количестве SELECT и JOIN запросов к таблицам.

На рисунке \ref{img:ui-relation} представлено меню миграции отношения.
\includeimage{ui-relation}{f}{h}{\textwidth}{Меню миграции отношения}

Далее происходит переход на этап определение кратности отношения, 
от кратности отношения будет зависеть то, какой тип будет иметь поле в JSON-документе.

\clearpage

На рисунке \ref{img:ui-reference} представлено меню ввода кратности отношения.
\includeimage{ui-reference}{f}{h}{\textwidth}{Меню ввода кратности отношения}


В случае, если отношение реализуется вложенным документом, 
необходмо произвести аналогичную процедуру миграции для вложенного документа.
Когда все таблицы мигрированы происходит возврат в интерактивную среду.

Для того чтобы завершить процесс миграции следует ввести команду \textbf{migration finish}.
При вводе этой команды на экран будет выведена сформированная очередь и будет начат процесс миграции данных
в соответствии с сформированной на предыдущем этапе схемой.

На рисунке \ref{img:ui-finish} представлен вывод команды \textbf{migration finish}.
\includeimage{ui-finish}{f}{h}{\textwidth}{Вывод сформированной очереди}

По завершении этой команды в MongoDB появятся мигрированные данные из PostgreSQL.
